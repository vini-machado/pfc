% ----------------------------------------------------------
% Introdução (exemplo de capítulo sem numeração, mas presente no Sumário)
% ----------------------------------------------------------
\chapter{Introdução}
% ----------------------------------------------------------

No início do século XX, boa parte da comunidade científica estava empenhada em discutir
aspectos relacionados à utilização da energia atômica. No contexto da Segunda Guerra Mundial
e frente à ameaça da influência nazista na Europa, cientistas como Arthur Compton e
Volney C. Wilson, tomados por um sentimento de dever patriótico, fizeram parte do grupo
de pesquisa sobre fissão nuclear organizado pelo governo dos Estados Unidos da América,
o Projeto Manhattan, cujo objetivo era desenvolver a primeira bomba atômica do mundo,
uma arma poderosa o suficiente para mudar o curso da guerra \cite{Badash2005}.

Após a detonação das bombas de Hiroshima e Nagasaki, as tensões geopolíticas em todo planeta foram elevadas,
dada a existência e o poder de uma nova categoria de armamento de destruição em massa, as bombas de fissão nuclear.
Por esse motivo, diversos países iniciaram uma corrida pela construção de armas nucleares, de forma a conseguir
garantir sua própria segurança e poderio militar. \cite{Gavin2010}

